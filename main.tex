\documentclass{IOS-Book-Article}

\usepackage{mathptmx}
\usepackage{soul}\setuldepth{article}
%\usepackage{times}
%\normalfont
%\usepackage[T1]{fontenc}
%\usepackage[mtplusscr,mtbold]{mathtime}
\def\hb{\hbox to 11.5 cm{}}

\begin{document}

\pagestyle{headings}
\def\thepage{}

\begin{frontmatter}      

    \title{Experimental study of un-stationary tip vortex cavitation}
    \markboth{}{June 2025\hb}

    \author[A]{\fnms{Giovanni} \snm{Franzosi}
    \thanks{Corresponding Author: Giovanni Franzosi, giovanni.franzosi@edu.unige.it.}},
    \author[A]{\fnms{Afaq Ahmed} \snm{Abbasi}}
    \author[A]{\fnms{Michele} \snm{Viviani}}
    and
    \author[A]{\fnms{Giorgio} \snm{Tani}}

    % \author[A]{\fnms{Giovanni} \snm{Franzosi}\orcid{0009-0000-8420-1034}
    % \thanks{Corresponding Author: Giovanni Franzosi, giovanni.franzosi@edu.unige.it.}},
    % \author[A]{\fnms{Afaq Ahmed} \snm{Abbasi}\orcid{0000-0002-8691-2451}}
    % \author[A]{\fnms{Michele} \snm{Viviani}\orcid{0000-0001-6212-1586}}
    % and
    % \author[A]{\fnms{Giorgio} \snm{Tani}\orcid{0000-0002-9471-6338}}
    
    \runningauthor{G. Franzosi et al.}

    \address[A]{University of Genoa, Department of Naval Architecture, Electric, Electronic and Telecommunication Engineering, Via Montallegro 1, Genova, 16145, Italy}

    \begin{abstract}

        This study presents an experimental approach to studying the dynamics of tip vortex cavitation. The selected case study is a four-bladed model-scale controllable pitch propeller tested under specific flow and operational conditions.

        The proposed method integrates high-speed video recordings with advanced computer vision techniques to quantitatively assess the characteristics of cavitating vortex dynamics, focusing in particular on the variation of vortex volume as a function of the propeller blade's angular position. The results demonstrate that the proposed approach enables precise measurements of vortex dynamics, offering valuable insights for future investigations. 
        
        Specifically, direct access to these measurements may facilitate the study of how vortex cavitation dynamics influence the spectral characteristics of radiated noise. Such insights are crucial for developing and validating underwater noise spectrum predictive models that comprehensively describe the noise sources.
        
    \end{abstract}

    \begin{keyword}
        Experimental Hydrodinamics \sep Tip Vortex Cavitation \sep  Model Scale Propeller \sep  Computer Vision
    \end{keyword}

\end{frontmatter}

\markboth{June 2025\hb}{June 2025\hb}

\section{Introduction}


\end{document}