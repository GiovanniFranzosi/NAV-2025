\documentclass{IOS-Book-Article}

\usepackage{mathptmx}
\usepackage{soul}\setuldepth{article}
\usepackage{graphicx}
\usepackage{caption}
\usepackage{amsmath}
\usepackage{subfig}

\def\hb{\hbox to 11.5 cm{}}

\begin{document}

\pagestyle{headings}
\def\thepage{}

\begin{frontmatter}      

    \title{Experimental study of un-stationary tip vortex cavitation}
    \markboth{}{June 2025\hb}   

    \author[A]{\fnms{Giovanni} \snm{Franzosi}
    \thanks{Corresponding Author: Giovanni Franzosi, giovanni.franzosi@edu.unige.it.}},
    \author[A]{\fnms{Afaq Ahmed} \snm{Abbasi}}
    \author[A]{\fnms{Michele} \snm{Viviani}}
    and
    \author[A]{\fnms{Giorgio} \snm{Tani}}

    % \author[A]{\fnms{Giovanni} \snm{Franzosi}\orcid{0009-0000-8420-1034}
    % \thanks{Corresponding Author: Giovanni Franzosi, giovanni.franzosi@edu.unige.it.}},
    % \author[A]{\fnms{Afaq Ahmed} \snm{Abbasi}\orcid{0000-0002-8691-2451}}
    % \author[A]{\fnms{Michele} \snm{Viviani}\orcid{0000-0001-6212-1586}}
    % and
    % \author[A]{\fnms{Giorgio} \snm{Tani}\orcid{0000-0002-9471-6338}}
    
    \runningauthor{G. Franzosi et al.}

    \address[A]{University of Genoa, Department of Naval Architecture, Electric, Electronic and Telecommunication Engineering, Via Montallegro 1, Genova, 16145, Italy}

    \begin{abstract}

        This study presents an experimental approach to studying the dynamics of tip vortex cavitation. The selected case study is a four-bladed model-scale controllable pitch propeller tested under specific flow and operational conditions.

        The proposed method integrates high-speed video recordings with advanced computer vision techniques to quantitatively assess the characteristics of cavitating vortex dynamics, focusing in particular on the variation of vortex volume as a function of the propeller blade's angular position. The results demonstrate that the proposed approach enables precise measurements of vortex dynamics, offering valuable insights for future investigations. 

        Specifically, direct access to these measurements may facilitate the study of how vortex cavitation dynamics influence the spectral characteristics of radiated noise. Such insights are crucial for developing and validating underwater noise spectrum predictive models that comprehensively describe the noise sources.

    \end{abstract}

    \begin{keyword}
        Experimental Hydrodinamics \sep Tip Vortex Cavitation \sep  Model Scale Propeller \sep  Computer Vision
    \end{keyword}

\end{frontmatter}

\markboth{June 2025\hb}{June 2025\hb}

\section{Introduction}

    Tip vortex cavitation is one of the most significant hydrodynamic phenomena affecting the operation of a marine propeller. This form of cavitation typically onsets earlier than other cavitation types, resulting in a frequent occurrence even at low speed or for propellers suitably designed to limit cavitation.
    The widespread presence of tip vortex cavitation poses significant challenges, as it is widely recognized as one of the primary sources of underwater radiated noise (URN).
    Understanding and predicting the effects of vortex cavitation and its dynamics on the URN spectrum is essential for naval engineering and propeller design.

    Despite this, the available tools for estimating the radiated noise spectrum during the design phase remain limited. 
    From a numerical perspective, noise prediction often relies on complex, computationally expensive methods based on the Ffowcs Williams–Hawkings equations~\cite{Ffowcs1969}, generally yielding accurate results. These methods~\cite{Testa2018} effectively reproduce noise generation and propagation but demand high-resolution modelling of turbulence and cavitation, often unfeasible with standard grids and models. As a result, their use in early design stages is impractical, limiting them to research or final propeller validation.
    Nowadays the most commonly adopted approaches for the prediction of propeller noise rely on experimental measurements on small scale propeller models ~\cite{Tani2019}.
    Experimental testing is the most reliable and accurate method for noise prediction, however, similarly to numerical methods, its application during the early stages of design is not economically feasible.

    For this reason, more straightforward and cost-effective approaches have been developed over time to provide noise estimates with acceptable accuracy while minimizing both costs and time requirements. 
    The scientific literature offers various examples of these methods that can be grouped into two main categories: semi-empirical and data-driven.
    Both semi-empirical and data-driven methods extensively rely on experimental data to establish robust correlations between propeller geometries, operating conditions, and the spectrum of radiated noise. 

    However, semi-empirical methods incorporate experimental data to refine and validate physics-based models~\cite{Bosschers2018}. Whereas data-driven methods exploit knowledge of phenomena from experimental datasets to train predictive algorithms~\cite{Miglianti2020}.
    Although these approaches are considerably different, both can significantly benefit from a deeper physical understanding of the phenomenon. Therefore, developing experimental methodologies to ensure accurate and detailed measurements of vortex cavitation is essential.

    In this context, experimental approaches based on computer vision techniques can provide a significant contribution. Similar methods have already been successfully applied to the study of cavitation in the past~\cite{Pereira1998},~\cite{Savio2011}.
    More recently, computer vision techniques have been combined with the temporal resolution offered by high-speed cameras to analyze erosive cavitation~\cite{Franzosi2023},~\cite{Franzosi2025}. 
    The quantitative measurements enabled by these approaches appear highly promising in the study of vortex cavitation.

    This study applies computer vision approaches to the investigation of tip vortex cavitation. The proposed method has been applied to a four-bladed model-scale marine propeller. The propeller was tested under specific operating conditions and in a non-uniform flow, which enabled the observation of non-stationary tip vortex cavitation dynamics. Combining advanced computer vision techniques with information from high-speed recordings allows assessing cavitation volume and its fluctuations over the propeller rotation.

    The results prove that the proposed method enables precise and detailed measurements of the tip vortex dynamics, which are challenging to access with other experimental techniques. 
    The availability of these data provides valuable insights for future investigations. Specifically, direct access to these measurements could enable the study of how vortex cavitation dynamics influence the spectral characteristics of the radiated noise.
    These insights pave the way for a deeper understanding of the phenomenon and the development or validation of predictive models of the underwater noise spectrum, aiming to comprehensively describe noise sources.

\section{Measurements Setup}

    The experimental campaign was conducted at the University of Genoa's Cavitation Tunnel. The case study selected for the experiments is a four-bladed model-scale controllable pitch propeller. The propeller has a diameter of $0.23 m$, a pitch ratio at $0.7R$ of 1.3, and an expanded blade area ratio $A_E/A_O = 0.7$. The propeller rotation rate was fixed at $n=18 Hz$, adjusting the flow velocity to obtain a thrust coefficient $K_T=0.20$.
    The cavitation test was conducted applyng vacuum at the tunnel. In this case the experiments were conducted keeping the cavitation number constant at $\sigma_{07R}=2.90$ wich corresond to a ratio $\sigma/\sigma_i=0.48$.
    Since cavitation dynamics are very sensitive to water quality, accurately monitoring this parameter is necessary to ensure the reliability of the experiments. In this regard, dissolved oxygen levels were measured, being approximately $4.4 ppm$ for these tests.
    A device known as a wake screen was employed to achive a specific inflow condition at the propeller disk. In this case, the wake screen was designed to produce a highly non-uniform flow. The wake at the propeller disk is characterized by a strong wake peak and high velocity gradient (Figure~\ref{fig:Wake}). This kind of inflow condition is rather common in real applications.

    \begin{figure}[htbp]
        \centering
        \includegraphics[width=0.5\textwidth]{FIGURE/Wake.png}
        \caption{Wake screen design and inflow condition at the propeller disk.}
        \label{fig:Wake}
    \end{figure}

    The cavitation tunnel was equipped with three high-speed cameras were employed in a multi-camera configuration. Two cameras (SpeedSense LAB 340CM 12M-70) were positioned to observe the propeller from above, while the third (Phantom VEO710L) was placed laterally. The experimental setup was equipped with a synchronizer (DANTEC Performance Synchroniser), enabling precise synchronization between the cameras and the propeller's rotation. This ensured consistent blade positioning across revolutions. 
    The three cameras operated at a shared acquisition frequency of $1134 \ FPS$, achieving a spatial resolution of approximately $5^\circ$.
    The scene was illuminated by two high-intensity LED stroboscopic lights (GSVITEC Multiled), providing uniform and adequate lighting. 

    To measure the volume of cavitating vortexes a specific computer vision technique was selected. The chosen techniques is the known as ``shape by silhouette'' which deduce the three-dimensional shape of an object from the silhouettes projected in the images. The camera positioning was chosen, within the constraints of the tunnel optical access, to maximize the diversity of observation angles. To further diversify the observation points, the cameras were significantly tilted relative to the tunnel windows.
    This choice improved significantly the results of the shape by silhouette. 

    Applying computer vision techniques requires introducing a mathematical model to describe cameras. A camera model defines the mathematical relationship that maps real-world coordinates ($m$) to image pixel coordinates ($M$).
    In this study, we utilize the pinhole camera model described by Equation~\ref{eq:pinhole}.

    \begin{equation}
        m = K \cdot \begin{pmatrix} I & 0 \end{pmatrix} \cdot \begin{pmatrix} R & T \\ 0 & 1 \end{pmatrix} \cdot M
        \label{eq:pinhole}
    \end{equation}

    In this equation, $R$ and $T$ are the extrinsic parameters, representing the rotation matrix and translation vector that relate the physical-world coordinates to the camera reference frame. The matrix $K$, known as the camera matrix, contains the intrinsic parameters of the model. Specifically, $K$ takes the form shown in Equation~\ref{eq:cameraMat}.

    \begin{equation}
        K = \begin{pmatrix}
            f_x & 0 & c_x \\
            0 & f_y & c_y \\
            0 & 0 & 1
        \end{pmatrix}
        \label{eq:cameraMat}
    \end{equation}

    Here, $f_x$ and $f_y$ are the camera's focal lengths along the two axes, while $c_x$ and $c_y$ are the pixel coordinates of the camera's principal point.

    Despite its simplicity, this camera model offers several advantages, including high accuracy and reliability. However, the pinhole model presents some limitations, especially when adopted in unconventional environments such as a cavitation tunnel. The model assumes that perspective rays are straight. In this case, the air-plexiglass-water interface at the tunnel's window causes light refraction, invalidating this assumption.
    Additionally, the pinhole model describes a lensless camera, whereas practical applications require lenses to obtain high-quality images. 
    Nevertheless, lenses introduce optical distortions that can affect measurement accuracy. In addition to lenses, Scheimpflug devices are often incorporated into the setup for such applications. These devices enhance depth of field by tilting the focal plane, allowing both near and distant elements to remain in focus. Similar to lenses, Scheimpflug systems also introduce optical distortions in the acquired images.

    Adopting appropriate solutions to mitigate these issues is essential for ensuring the accuracy of measurements. The literature presents specific mathematical models designed to describe and correct the primary sources of distortion. 
    The available distortion models were combined into nine distinct configurations to select the most appropriate one for accurately describing the optical distortion in this case.

    In addition to correcting for lens-induced distortions, using these models partially compensates for light refraction effects. However, to achieve precise measurements, a plexiglass prism must be added in front of each camera, ensuring that the interface is parallel to the camera sensor. This setup significantly reduces the estimation error on camera parameters due to the refraction effects, improving the accuracy of the measurement.

    
\newpage
   
\section{Computer Vision Measurements}

\section{Results Analysis} 

\section{Conclusions}

\newpage

\bibliographystyle{vancouver} 
\bibliography{bibliography} 

\end{document}




    
    
    
    % In questo caso sono state considerate quattro sorgenti di distorsione principali: distorsione radiale, distorsione tangenziale, distorsione thin-prrism e distorsione di rotazione.
    
    


    % However, lenses, which are essential in practical applications, introduce optical distortions that can affect measurements.  
    %         In addition to lenses, it is good practice to incorporate Scheimpflug devices into the setup for these applications. These devices enhance depth of field by tilting the focal plane, allowing both near and distant elements to remain in focus. However, Scheimpflug systems also introduce optical distortion into the acquired images.
    
    
    % In questo caso è stato montato un prisma di plexiglass di fronte a ciascuna telecamera, garantendo che l'interfaccia ottica rimanesse parallela ai sensor
    
    
    
    
    % Questo comporta distorsioni ottiche che possono influenzare i risultati delle misure. 
    
    
    
    % Per mitigare questi problemi, è stato montato un prisma di plexiglass di fronte a ciascuna telecamera, garantendo che l'interfaccia ottica rimanesse parallela ai sensori della telecamera.
    
    
    
    
    
    
    % Uno dei principali problemi è la presenza di distorsioni ottiche. Per mitigare questi problemi, è stato montato un prisma di plexiglass di fronte a ciascuna telecamera, garantendo che l'interfaccia ottica rimanesse parallela ai sensori della telecamera. 
    
    
    
    % Tuttavia, il modello presenta anche dei limiti, come la presenza di distorsioni ottiche. Per mitigare questi problemi, è stato montato un prisma di plexiglass di fronte a ciascuna telecamera, garantendo che l'interfaccia ottica rimanesse parallela ai sensori della telecamera.





    % % INTRODUCI ANCORA IL MODELLO I PROBLEMI DEL TUNNEL E LA CALIBRAZIONE.

    
    % However, this alignment posed potential measurement challenges,


    
    % the cameras were positioned within the constraints of optical access, significantly improving the results of shape-by-silhouette reconstructions. To further diversify the observation points, the cameras were significantly tilted relative to the tunnel windows. However, this alignment posed potential measurement challenges, as detailed in Chapter \ref{chap:chapter4}. To mitigate these issues, a plexiglass prism was mounted in front of each camera, ensuring that the optical interface remained parallel to the camera sensors.


% To maximize the diversity of observation angles, the cameras were positioned within the constraints of optical access, a choice that, as previously discussed, significantly improved the results of shape-by-silhouette reconstructions.
% To further diversify the observation points, the cameras were significantly tilted relative to the tunnel windows. However, this alignment posed potential measurement challenges, as detailed in Chapter \ref{chap:chapter4}. To mitigate these issues, a plexiglass prism was mounted in front of each camera, ensuring that the optical interface remained parallel to the camera sensors.

